%*******************************************************
% Einleitung
%*******************************************************

%\begingroup
%\let\clearpage\relax
%\let\cleardoublepage\relax
%\let\cleardoublepage\relax

%\vfill
%\pdfbookmark{Einleitung}{Einleitung}
\chapter{Einleitung}
\label{chp:einleitung}


\section{Einleitung}
Street-Artists lebten ihre k�nstlerische T�tigkeit meist unentgeltlich im Schutz der Nacht und auf der Flucht vor der Polizei aus. Ist es m�glich, Street-Art in einem geregelten Rahmen im Atelier zu t�tigen und sich damit den Lebensunterhalt zu finanzieren? Wo steht Street-Art auf dem Weg von den Strassen in den etablierten Kunstmarkt und damit ins Museum?
\\
Diesen Fragen ging ich in meiner Arbeit auf den Grund. Ich bewerkstelligte dies durch Interviews mit Street-Artists sowie intensive Recherche und Analyse von B�chern und Artikeln. \\
Am Schluss meiner Arbeit ziehe ich eine Bilanz wie die Street-Art-Szene momentan zum kommerziellen Kunstmarkt steht und wie sich das in naher Zukunft ver�ndern k�nnte.
\\
Zuerst werde ich jedoch den Term Street-Art untersuchen und erkl�ren, sowie die Herkunft und Entstehung dieser Kunstbewegung dem Leser n�her bringen.
Danach werde ich erl�utern warum man Street-Art zur angewandten Kunst z�hlen kann und im letzten Teil er�rtern, wie weit die Etablierung der Street-Art als anerkannte Kunstform fortgeschritten ist.
\\
\clearpage
\section{Definition des Begriffs Street-Art}
Der Begriff Street-Art entstand um das Jahr 2004 auf dem Woostercollective\footnote{Ein wichtiger Street-Art-Blog mit einem umfangreichen Archiv, \url{www.woostercollective.com}}. Um diese Zeit bestand der Bedarf nach einem Begriff f�r diese neue Ausdrucksform im �ffentlichen Raum. Street-Art ist das Wort, was sich schlussendlich durchsetzte, jedoch gab es mehrere Kandidaten f�r die Benennung dieses damals aufkommenden Ph�nomens.
\\%%
Es gibt zum Beispiel den Begriff Urban Art. Urban Art klingt legitimer und sauberer und wird auch bei Ausstellungen oft verwendet, um die Distanz zum Illegalen zu betonen. Doch das Wort bedeutet genau das gleiche: Kunst im st�dtischen Raum. Die Illegalit�t des Handelns ist bei Urban Art jedoch schw�cher konnotiert, da vom Begriff her kein ein so starker Zusammenhang mit Graffiti besteht wie bei Street-Art. Denn Graffiti und Street-Art worden oft im selben Atemzug genannt. Da ich jedoch den unautorisierten und Graffiti-nahen Charakter von Street-Art in meiner Arbeit nicht verstecken, sondern offen behandeln will, werde ich diesen Begriff nicht prim�r verwenden. Viele Galerien und Ausstellungen benutzen jedoch den Begriff Urban Art, eben um die Kunst von Wandschmierereien abzugrenzen und sie so greifbarer f�r die Masse zu machen.
\\%%
Ein anderer Begriff der in Frage kommt, ist Post-Graffiti. Das Wort impliziert die N�he zu, Graffiti doch grenzt sich gleichzeitig davon ab. Auch ist der Begriff aussagekr�ftiger, w�hrend Street-Art buchst�blich Strassenkunst heisst und damit das (eigentlich gemeinte) Anbringen von Schablonenkunstwerken, Cut-outs\footnote{Auf Papier gezeichnete oder gesprayte und dann ausgeschnitten und aufgeklebte Kunstwerke} und Stickern in einer st�dtischen Umgebung genauso beschreibt wie Kreidezeichnungen von Kindern  auf der Strasse und sogar jonglierende Strassenk�nstler,  bezeichnet Post-Graffiti eher einen Zeitraum. Und zwar den Zeitraum nach Graffiti, denn Street-Art in seiner heutigen Form entstand zirka Mitte der 1980er, wogegen Graffiti schon in den 1970ern sein Unwesen trieb.
\\%%
Die Subkultur der Street-Art ging aus der Graffiti-Szene hervor. Aus den Writer-Crews spalteten sich einzelne K�nstler ab und konzentrierten sich auf das Aufkleben von Stickern und Paste-Ups\footnote{Mit Kleister an die W�nde geklebte Plakate, die meist bemalt oder besprayt sind.}, sowie auf das Sprayen ihrer Motive mithilfe von Schablonen. Im Gegensatz zu den Writern platzierten die Street-Artists meist nicht ihren Namen auf die �ffentlichen W�nde, sondern eine Vielzahl von Motiven, darunter politische Themen, Anti-Werbung oder einfach  �sthetische Versch�nerungen der grauen Stadt.  
Doch auch Post-Graffiti  ist kein Idealer Begriff. Denn der Wortteil "`post-"' impliziert dass Graffiti nicht mehr existiert, die Ausdrucksweise Post-Graffiti schliesst eigentlich eine Weiterexistenz von �blichem Graffiti aus, und das ist keineswegs der Fall.\\
Der Begriff Street-Art hat sich durchgesetzt, da er wortw�rtlich nur impliziert, dass es Kunst auf der Strasse ist. Darunter f�llt auch die unautorisierte Kunst im �ffentlichen Raum, und das ist ja die Hauptsache. Ein Begriff sollte prim�r einfach mal unmissverst�ndlich das beschreiben was er bedeutet, und das ist bei Street-Art der Fall. John Fekner, eine einflussreiche Figur der Street-Art-Bewegung,  definiert Street-Art als: "` all art on the street that�s not graffiti."'\footnote{\cite[S.14]{lewisohn2008}}. Gemeint ist damit sowohl das Platzieren von Schablonen-Werken, Skulpturen, sogenannten Wheatpastes, Stickern als auch Kunst-Interventionen, Guerilla Kunst, Videoprojektionen und Strasseninstallationen.\\
Etwas passt jedoch nicht in Fekners Aussage. Denn obwohl es im Begriff selbst vorkommt, ist Street-Art oft gar keine Kunst, sondern vielmehr eine Ausdrucksform im urbanen Raum. In den Worten des Basler Street-Art-K�nstlers Bustart:
\begin{quote}
"`\emph{Mir pers�nlich passt der Name "`Street-Art"' �berhaupt nicht, denn dieser nennt jede Ver�nderung im �ffentlichen Raum Kunst, was in meinen Augen nicht unbedingt stimmt. Street-Art kann fr�hlich sein, traurig, politisch, lustig oder auch nichtssagend, in den seltensten f�llen ist es Kunst. Ich finde dies h�ngt nur vom Betrachter ab und seiner Einstellung gegen�ber dieser "`Kunst"'}\footnote{Bustart, im E-Mail Interview vom 28.3.12}
\end{quote}