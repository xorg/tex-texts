%************************************************
\chapter{Schlusswort}
\label{chp:schlusswort}
%************************************************

Zusammenfassend kann man sagen, dass Street-Art noch nicht im etablierten Kunstbetrieb angekommen ist. Dies ist darauf zur�ckzuf�hren, dass die Street-Art-Szene sich nicht aus traditionellen Kunstformen herausgebildet hat, sondern aus einer Subkultur stammt, die die Kapitalformen anders gewichtet als der etablierte Kunstbetrieb. Dazu kommt, dass die Street-Art als angewandte Kunst gilt, und auch deshalb vom Kunstbetrieb nicht anerkannt wird. Auch das Verhalten und der Habitus der Street-Art-Bewegung ist mit der hochkulturellen Kunstwelt nicht kompatibel. Es gibt zwar Ans�tze, mit Ausstellungen und Galerien Street-Art-K�nstler in den kommerziellen Kunstmarkt zu bef�rdern, dies ist jedoch sehr schwierig zu bewerkstelligen ohne dass man die Street-Art und damit die illegale Kunst im �ffentlichen Raum verl�sst. Doch Banksy zeigt, dass es m�glich ist, den Spagat zwischen Strasse und Galerie zu meistern. Es bleibt nun abzuwarten ob es weitere K�nstler schaffen werden. Die Bewegung der Street-Art ist sehr lebendig und es entstehen jeden Tag tausende neue Werke im �ffentlichen Raum. Auch die Street-Art-Ausstellungen vermelden steigende Besucherzahlen. Ich bin gespannt wie sich dies n�chster Zeit entwickeln wird, denn Street-Art hat definitiv das Potential, eine neue kunsthistorische Epoche zu werden.



