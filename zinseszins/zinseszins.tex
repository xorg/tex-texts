% Environment
\environment short_text


\starttext

\title{Der Zinseszins}

\subject{Was ist Zinseszins?}
Als Zinseszins wird allgemein {\em Zins auf Zinsen} bezeichnet.

Zinseszins entsteht beispielsweise bei Geldanlagen, bei denen Zinsbeträge dem bisherigen Kapital zugeschlagen werden und somit in allen Folgeperioden mitverzinst werden.\footnote{Quelle: http://www.zinsen-berechnen.de/online-rechner/zinseszins.php 26.01.10} 
Zinseszins entsteht eigentlich bei allen Sparkonten auf Banken oder Darlehensschulden bei Firmen. Wo genau Zinseszins verwendet wird und wo nicht, erkläre ich weiter unten.

Der Zinseszins-Effekt ist oft bei Schulden gefährlich, weil dort die Schulden zuerst langsam steigen, dann aber schnell ausser Kontrolle geraten können, da die Zinsen beim Zinseszins {\em exponentiell} steigen.

\subject{Ein Beispiel}
Nehmen wir an Hans M. legt 1000 Franken bei der Bank zu einem Zinssatz von 2\% mit Zinseszins an. Der Zins wird jährlich abgerechnet.


Ende des ersten Jahres erhält er auf sein Sparkonto 20 Franken gutgeschrieben. Der Betrag auf dem Konto beträgt jetzt 10'020 Franken.

Ende des zweiten Jahres werden nun zusätzlich zu den 20 Fr. noch 2\% von den 20 Fr. zugeschlagen, das neue Kapital beträgt also 1040.40 Franken.

\subject{Die Formel}
Wenn man  herausfinden will auf welches Endkapital ein Anfankskapital mit einem konkreten Zinssatz nach einer bestimmten Anzahl von Auszahlungszeiträumen (üblicherweise ein Jahr) angewachsen ist, geht das am schnellsten mit folgender Formel:

\startformula
K_n = K_0(1+\frac{z}{100})^n
   $K_n$ = Endkapital; $K_0$ = Anfangskapital; $z$ = Zinssatz; $n$ = Anzahl der geltenden Zeiträume/Jahre.
\stopformula
Die Formel rürt daher, dass zuerst der Zins für ein Jahr ausgerechnet wird, mit dem anfänglichen Startkapital addiert wird, und dann das enstandene (Zwischen)Endkapital des ersten Jahres als Anfangskapital für das nächste Jahr verwendet wird. Also $K_0(1+\frac{z}{100})(1+\frac{z}{100})=K_0(1+\frac{z}{100})^2$ was dann $K_n = K_0(1+\frac{z}{100})^n$ ergibt.\footnote{Quelle: http://de.wikipedia.org/wiki/Zinseszins 01.03.10}



Wenden wir einmal diese Formel auf unser Beispiel von vorher an.
\startformula
K_2_5=1'000(1+\frac{2\%}{100})^5^0=2691.60
\stopformula
Wenn unser Hans also das Geld 25 Jahre mit festem Zinssatz anlegt, steigt sein Kapital also auf {\em 2691.60}. Der Zinseszins entfaltet seine volle Wirkung also erst bei  {\em längeren Laufzeiten.}

{\em Eine Anmerkung: Der effektive Zinssatz verändert sich leicht wenn man mehrmals pro Jahr abrechnet.} 

\subject{Unterschied zwischen einfachem Zins und Zinseszins}
Wie ihr vielleicht schon gemerkt habt, ist die Besonderheit des Zinseszins die, dass er nicht linear steigt, sondern exponentiell.

Das kommt daher dass bei der einfachen Verzinsung der Zinsbetrag des Anfangkapitals jedes Jahr addiert wird und sich nicht verändert. Also wenn man ein Startkapital von 1000 Franken zu einem Zinsatz von 5\% linear verzinst, ergibt das bei 5 Jahren Laufzeit ein Endkapital von 1500.- .

Einfache Verzinsung wird eigentlich nur bei private Geldgeschäften verwendet, da dort Zinseszinsen verboten sind. Überall, wo Firmen im Spiel sind, werden heutzutage Zinseszinsen verwendet.

Hier noch ein kleiner Vergleich der Entwicklung eines Kapitals mit linearer und exponentieller Verzinsung.
\placefigure[fig:bla]{Vergleich von einfachem Zins und Zinseszins bei 100.- Startkapital und 9\% Zinsen}{\externalfigure[comparison][scale=40, factor=fit, method=pdf]}

\subject{Aufgaben}
\startitemize[n]

\item Bausparer Bruno will Geld für ein neues Haus anlegen. 
Dazu stehen ihm 150'000.- zur Verfügung.
Seine Bank macht ihm zwei Angebote: 

a.) Das Geld wird mit linearer Verzinsung zu einem festen Zinssatz von 3\% angelegt.

b.) Ein weiteres Angebot der Bank ist, das Geld zu einem 2.5\%igen Zinseszins anzulegen.

Bruno hat 25 Jahre Zeit zu sparen. (Tipp: Die Formel für lineare Verzinsung lautet: $K_0(1+\frac{z}{100}n)$)

Welches Angebot lohnt sich mehr?

\item Susi hat in 13 Jahren ihr Kapital mit einem Zinssatz von 5\% auf 6384.30 Fr. erhöht.

Wieviel Geld hatte Susi vor diesen 13 Jahren auf dem Konto? (Das Geld wurde mit Zinseszins verzinst)
\stopitemize
\stoptext
