% Environment
\environment short_text


\starttext

\title{Der Zinseszins}

\subject{Was ist Zinseszins?}
Als Zinseszins wird allgemein {\em Zins auf Zinsen} bezeichnet.

Zinseszins entsteht beispielsweise bei Geldanlagen, bei denen Zinsbeträge dem bisherigen Kapital zugeschlagen werden und somit in allen Folgeperioden mitverzinst werden.\footnote{Quelle: http://www.zinsen-berechnen.de/online-rechner/zinseszins.php 26.01.10} 

Der Zinseszins-Effekt wird oft bei Kreditinstituten verwendet um die Gläubiger zu möglichst raschem Zahlen der Schulden zu drängen, da die Zinsen beim Zinseszins {\em exponentiell} steigen.

Am Ende der Abrechnungsperiode (üblicherweise ein Jahr) wird  der fester Zins auf den bereits angefallenen Zins berechnet, und dieser dann dem Kapital zugeschlagen.
Dadurch steigen die Zinsbeträge nach gewisser Zeit viel schneller als ohne  Zinseszins. Das kann bei Zinsen auf Geldanlagen nützlich sein, ist aber auch eine grosse Gefahr bei Schulden.


\subject{Ein Beispiel}
Nehmen wir an Hans M. legt 10'000 Franken bei der Credit Suisse zu einem Zinssatz von 5\% mit Zinseszins an. Der Zins wird jährlich abgerechnet.


Ende Jahres erhält er auf sein Sparkonto 500 Franken gutgeschrieben. Der Betrag auf dem Konto beträgt jetzt 10'500 Franken.

Dieser Zinsbetrag von 500 Franken erhöht sich aber gleichzeitig um diese 5\%. Das wären 25 Franken. Nächstes Jahr werden also statt 500 Franken 525 Franken gutgeschrieben, nach zwei Jahren 551.25 Franken und so weiter. Es wird jeweils nur der Zins verzinst und nicht das ganze Kapital. Der Zins wird also Ende des Jahres aus dem Kapital entnommen, verzinst und dem Kapital wieder hinzugefügt.

\subject{Die Formel}
Wenn man aber herausfinden will auf welches Endkapital ein Anfankskapital mit einem konkreten Zinssatz nach einer bestimmten Anzahl von Auszahlungszeiträumen (üblicherweise ein Jahr) angewachsen ist, geht das am schnellsten mit folgender Formel:

\startformula
K_n = K_0(1+\frac{z}{100})^n
   $K_n$ = Endkapital; $K_0$ = Anfangskapital; $z$ = Zinssatz; $n$ = Anzahl der geltenden Zeiträume/Jahre. \footnote{Quelle: http://de.wikipedia.org/wiki/Zinseszins 01.03.10}
\stopformula
Wenden wir einmal diese Formel auf unser Beispiel von vorher an.
\startformula
K_2_5=10'000(1+\frac{5\%}{100})^2^5=33'863.55
\stopformula
Wenn unser Hans also das Geld 25 Jahre mit dem gleichen Zinssatz auf der Bank lässt, steigt sein Kapital auf 33'863.55 Franken um knapp 240\%, wohingegen der sich der Betrag bei gewöhnlichem Zins nur um 125\% auf 22'250 erhöht.  Der Zinseszins entfaltet seine volle Wirkung also erst bei {\em längeren Laufzeiten.}

{\em Eine Anmerkung: Der effektive Zinssatz verändert sich leicht wenn man mehrmals pro Jahr abrechnet.} 

\subject{Unterschied zwischen einfachem Zins und Zinseszins}
Wie ihr vielleicht schon gemerkt habt, ist die Besonderheit des Zinseszins die, dass er nicht linear steigt, sondern exponentiell.
Der einfache (auch linearer Zins genannte) Zins hingegen steigt immer um den gleichen Betrag, also Linear.

Dabei wird bei beiden Verzinsungen davon ausgegangen, dass der Zins am Ende der Verrechnungsperiode vom Kapital herausgenommen wird, dann das Kapital (bzw. der Zins beim Zinseszins) verzinst wird und der ganze Zins wieder dem Kapital zugefügt wird. 

\placefigure[fig:bla]{Vergleich von einfachem Zins und Zinseszins bei 100.- Startkapital und 9\% Zinsen}{\externalfigure[comparison2][scale=40, factor=fit, method=eps]}

\subject{Aufgaben}
\startitemize[n]

\item Bausparer Bruno will Geld für ein neues Haus anlegen. 
Dazu stehen ihm 150'000.- zur Verfügung.
Seine Bank macht ihm zwei Angebote: 

a.) Das Geld wird mit linearer Verzinsung zu einem festen Zinssatz von 3\% angelegt.

b.) Ein weiteres Angebot der Bank ist, das Geld zu einem 2.5\%igen Zinseszins anzulegen.

Bruno hat 25 Jahre Zeit zu sparen. (Tipp: Die Formel für lineare Verzinsung lautet: $K_0(1+\frac{z}{100}n)$)

Welches Angebot lohnt sich mehr?

\item Susi hat in 13 Jahren ihr Kapital von 2463.50 Fr. auf 6384.30 Fr. erhöht.

Bei welchem Zinssatz wurde das Geld angelegt? (Das Geld wurde mit Zinseszins verzinst)
\stopitemize
\stoptext
