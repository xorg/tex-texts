\setupcolors[state=start]   
\definecolor[headingcolor][r=0.7,b=0.2, g=0.1],
\definecolor[subjectcolor][r=0.6,b=0.4, g=0.2],
\setupheader[style=\rm\sl]
\setupheadertexts[Mathematik\hfill Stefan Schneider]
\setupfooter[style=\rm\sl]
\setupfootertexts[Mathematik\hfill Stefan Schneider]
%\setuppagenumbering[location={header,right}, style=bold]
%\setupitemize[inbetween={}, style=bold]
\setuphead[section,chapter][color=headingcolor]
\setuphead[subject][color=subjectcolor]
\setuphead[subject][style={\rm\tfb\bf}]
\setuphead[title][style={\rm\tfd\bf}]
\setupbodyfont[sans]
\definetypeface [boldmath] [mm] [boldmath] [latin-modern] [default]

\enableregime	[utf-8]
\mainlanguage	[de]
\setuppapersize[A4] [A4]
\setuplayout[location=middle,
  topspace=0.6cm,
  width=middle,
  cutspace=2cm,
  rightmargindistance=0.4cm,
  leftmargindistance=0.4cm,
  backspace=2cm,
  height=fit,
  rightmargin=0cm,
  leftmargin=1cm,
  bottomspace=0.6cm,
  footer=1.8cm,
  header=1.8cm,
  setup=strict] 

\starttext

\title{Der Zinseszins}

\subject{Was ist Zinseszins?}
Als Zinseszins wird allgemein {\bf Zins auf Zinsen} bezeichnet.

Zinseszins entsteht beispielsweise bei Geldanlagen, bei denen Zinsbeträge dem bisherigen Kapital zugeschlagen werden und somit in allen Folgeperioden mitverzinst werden.\footnote{Quelle: http://www.zinsen-berechnen.de/online-rechner/zinseszins.php 26.01.10} 

Der Zins steigt also {\bf exponentiell}

Ende Jahres wird  ein fester Zins auf den bereits angefallenen Zins berechnet.

Dadurch steigen Zinsbeträge viel schneller als ohne  Zinseszins. Das kann bei Zinsen auf Geldanlagen nützlich sein, ist aber auch eine grosse Gefahr bei Schulden.

\subject{Wie wird der Zinseszins berechnet?}
Mithilfe der Zinseszinsrechnung kann man herausfinden, auf welches Endkapital  ${\bf K_n}$  ein Anfangskapital ${\bf K_0}$  nach ${\boldsymbol n}$  Zeiträumen (üblicherweise ein Jahr) angewachsen ist, wenn mit einem festen Zinssatz von ${\bf z}$  verzinst wird.
Die Formel lautet:



\startformula
\framed{K_n = K_0(1+z)^n}
\stopformula


\stoptext

